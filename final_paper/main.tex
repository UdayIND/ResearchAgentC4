\documentclass[11pt,a4paper]{article}
\usepackage[utf8]{inputenc}
\usepackage{amsmath}
\usepackage{amsfonts}
\usepackage{amssymb}
\usepackage{graphicx}
\usepackage{cite}
\usepackage{url}
\usepackage{booktabs}
\usepackage{float}
\usepackage[margin=1in]{geometry}

\title{[Agent will fill: Research Title]}
\author{Autonomous Research Agent}
\date{\today}

\begin{document}

\maketitle

\begin{abstract}
[Agent will fill: Abstract summarizing the research problem, methodology, key results, and conclusions]
\end{abstract}

\section{Introduction}

[Agent will fill: Introduction section including:]
% - Problem statement and motivation
% - Research objectives
% - Brief overview of approach
% - Paper organization

\subsection{Background and Motivation}
[Agent will describe the research problem and why it's important]

\subsection{Related Work}
[Agent will summarize relevant literature from papers_summary.md]

\subsection{Contributions}
[Agent will list the main contributions of this work]

\section{Methodology}

\subsection{Dataset Description}
[Agent will describe the dataset using information from dataset_overview.md]

\subsection{Data Preprocessing}
[Agent will detail preprocessing steps from preprocessing.py]

\subsection{Exploratory Data Analysis}
[Agent will summarize EDA findings and reference visualizations]

\subsection{Model Architecture}
[Agent will describe the chosen model(s) and rationale]

\subsection{Training Procedure}
[Agent will detail the training process, hyperparameters, etc.]

\subsection{Evaluation Metrics}
[Agent will specify how model performance was measured]

\section{Results}

\subsection{Experimental Setup}
[Agent will describe the experimental configuration]

\subsection{Performance Results}
[Agent will present quantitative results in tables and figures]

% Example table placeholder
\begin{table}[H]
\centering
\caption{Model Performance Comparison}
\begin{tabular}{@{}lcccc@{}}
\toprule
Model & Accuracy & Precision & Recall & F1-Score \\
\midrule
[Agent will fill with actual results] & & & & \\
\bottomrule
\end{tabular}
\label{tab:results}
\end{table}

% Example figure placeholder
\begin{figure}[H]
\centering
\includegraphics[width=0.8\textwidth]{../data_analysis/visualizations/[agent_will_specify_filename]}
\caption{[Agent will provide descriptive caption]}
\label{fig:example}
\end{figure}

\subsection{Analysis and Discussion}
[Agent will interpret results and discuss implications]

\section{Conclusion}

\subsection{Summary of Findings}
[Agent will summarize key findings and contributions]

\subsection{Limitations}
[Agent will acknowledge limitations of the study]

\subsection{Future Work}
[Agent will suggest directions for future research]

\section{Acknowledgments}
This research was conducted by an autonomous AI research agent using the OpenAPI Agent Development Kit (ADK) framework.

\begin{thebibliography}{99}
% [Agent will populate with references from literature review]
% Example format:
% \bibitem{example2024}
% Author, A. et al. (2024). 
% \textit{Title of Paper}. 
% Journal/Conference Name.

\end{thebibliography}

\end{document} 